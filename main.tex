\documentclass[]{spie}  %>>> use for US letter paper
%\documentclass[a4paper]{spie}  %>>> use this instead for A4 paper
%\documentclass[nocompress]{spie}  %>>> to avoid compression of citations

\renewcommand{\baselinestretch}{1.0} % Change to 1.65 for double spacing
 
\usepackage{amsmath,amsfonts,amssymb}
\usepackage{graphicx}
\usepackage[colorlinks=true, allcolors=blue]{hyperref}
\usepackage{glossaries}
\title{Simulating the Effects of Exozodiacal Dust in WFIRST CGI observations}

\author[a]{Ewan S. Douglas}
\author[b]{John Debes}
\author[a]{Kian Miliani}
\affil[a]{University of Arizona, Tucson, AZ, USA}
\affil[b]{STScI, Baltimore, MD, USA}

\authorinfo{Further author information: (Send correspondence to A.A.A.)\\A.A.A.: E-mail: aaa@tbk2.edu, Telephone: 1 505 123 1234\\  B.B.A.: E-mail: bba@cmp.com, Telephone: +33 (0)1 98 76 54 32}

% Option to view page numbers
\pagestyle{empty} % change to \pagestyle{plain} for page numbers   
\setcounter{page}{301} % Set start page numbering at e.g. 301
 
\begin{document} 
\maketitle

\begin{abstract}

\end{abstract}
\input{acronyms}
% Include a list of keywords after the abstract 
\keywords{Manuscript format, template, SPIE Proceedings, LaTeX}

\section{INTRODUCTION}\label{sec:intro}  % \label{} allows reference to this section

The \gls{WFIRST} \gls{CGI}\cite{spergel_wide-field_2015,noecker_coronagraph_2016} will image circumstellar environments, imaging circumstellar disks\cite{schneider_quick_2014,schneider_detection_2016} and giant exoplanets  \cite{marley_quick_2014,ygouf_data_2016-1,bailey_lessons_2018} at visible wavelengths and extreme ($<<10^{-7}$) flux ratios\cite{douglas_wfirst_2018,kasdin_wfirst_2018}.

Modeling the sensitivity of a system to exoplanets requires combining telescope and detector parameters with coronagraph performance and target star properties \cite{nemati_sensitivity_2017,savransky_exosims_2018}.
Detailed modeling of exoplanet imaging coronagraph performance requires end-to-end  modeling. Ruane et al.\cite{ruane_review_2018} reviewed the basic elements of coronagraph design and the publicly available design tools.
The \gls{WFIRST} instrument has been extensively modeled in the PROPER\cite{krist_overview_2015,krist_wfirst_2017,krist_wfirst_2018,zhou_high_2018} and FALCO\cite{riggs_fast_2018,sidick_fast_2018} libraries.

This preliminary work extends those models to small angles from the target star, within the \gls{CGI} \gls{IWA}.
The \gls{IWA} for CGI is typically defined as the ``field radius from the star at which the PSF core throughput is 50\% of its maximum''\cite{krist_numerical_2015}.
In our solar system, solar flux and zodiacal dust number density are both decreasing with separation from the Sun\cite{rowan-robinson_improved_2013,kennedy_exo-zodi_2014}; hence, we expect scattered light from exozodiacal dust to peak close to the star. This potentially places a significant source of flux inside the \gls{IWA} but far enough from the central star to be partially resolved.

Since the \gls{PSF} of the star is highly aberrated by the coronagraph near the inner working angle, simulation of images in this region must be field-angle dependent and  convolution of a representative \gls{PSF} will not recover the unique diffraction pattern of compact sources. 
Examples of the evolution of the \gls{HLC} PSF morphology and brightness as a function of target separation along the horizontal axis are shown if Fig. \ref{fig:psfs}.



\begin{figure}
    \centering
    \includegraphics[width=0.51\textwidth]{0PSF.png}
    \includegraphics[width=0.49\textwidth]{17PSF_1lambdaD.png}
    \includegraphics[width=0.49\textwidth]{34PSF_2lambdaD.png}
    \includegraphics[width=0.49\textwidth]{51PSF.png}
    \includegraphics[width=0.49\textwidth]{68PSF.png}
    \caption{\acrshort{HLC} \gls{PSF} including transmission at a different separations. Top: On-axis coronagraph dark hole. Middle row: approx. 1 and 2$\lambda/D$ PSF. Significant flux is transmitted from sources inside the nominal 1.5$\lambda/D$ \acrshort{IWA}. Bottom row: approx. 3 and 4 $\lambda/D$ PSF. }
    \label{fig:psfs}
\end{figure}
\begin{figure}[htbp]
    \centering
    \includegraphics[width=0.95\textwidth]{flow.png}
    \caption{Simulation flow for field dependant PSF simulations of debris disks.}
    \label{fig:flow}
\end{figure}




\begin{figure}[htbp]
    \includegraphics[width=0.95\textwidth]{Unknown-4.png}
    \caption{Exozodiacal dust simulation assuming all dust is outside the \gls{IWA} }
    \label{fig:nogap}
    \end{figure}

    \begin{figure}[htbp]

    \centering
    \includegraphics[width=0.95\textwidth]{Unknown-7.png}
    \caption{}
    \label{fig:gap}
\end{figure}


\section{Methods}
\label{sec:intro}  % \label{} allows reference to this section
Fig. \ref{fig:flow} shows our simulation procedure. 
We generate a simulated dust image in units of Jansky per pixel using Zodipic\cite{kuchner_zodipic_2012} or another simple optically-thin radiative transfer model.
Each pixel in the input scene is finely sampled, typically on a 3 mas grid.  
We build a composite image by multiplying the flux in the input pixel by the corresponding PSF array.
The PSF arrays are sampled from the 1D \gls{HLC} \gls{PSF} grid publicly shared on the IPAC website\footnote{http://wfirst.ipac.caltech.edu}.
Rotational symmetry is assumed and interpolation is  nearest neighbor since the input grid is finely sampled (3 mas inside 4 $\lambda/D$).

\begin{figure}
    \centering
    \includegraphics[width=0.51\textwidth]{0PSF.png}
    \includegraphics[width=0.49\textwidth]{17PSF_1lambdaD.png}
    \includegraphics[width=0.49\textwidth]{34PSF_2lambdaD.png}
    \includegraphics[width=0.49\textwidth]{51PSF.png}
    \includegraphics[width=0.49\textwidth]{68PSF.png}
    \caption{\acrshort{HLC} \gls{PSF} including transmission at a different separations. Top: On-axis coronagraph dark hole. Middle row: approx. 1 and 2$\lambda/D$ PSF. Bottom row: approx. 3 and 4 $\lambda/D$ PSF.}
    \label{fig:my_label}
\end{figure}

\section{Preliminary Results}

    %to do: write ipynb DOI
    %fix bib
    %state that the disk brightens by >10%
    % summarize future plans
    
\acknowledgments % equivalent to \section*{ACKNOWLEDGMENTS}   


The authors acknowledge valuable inputs from  Vanessa Bailey,  Brian Kern, John Krist, Hanying Zhou,   and the rest of the JPL CGI team.
Portions of this work were supported by the WFIRST Science Investigation team prime award \#NNG16PJ24C.
Portions of this work were supported by the Arizona Board of Regents Technology Research
Initiative Fund (TRIF).
This research made use of the \gls{MGHPCC} via MIT Research Computing and High Performance Computing (HPC) resources supported by the University of Arizona (UA) TRIF, UITS, and RDI and maintained by the UA Research Technologies department.


This research made use of community-developed core Python packages, including: Astropy\cite{the_astropy_collaboration_astropy_2013}, Matplotlib\cite{hunter_matplotlib_2007}, SciPy\cite{jones_scipy_2001}, 
the IPython Interactive Computing architecture \cite{perez_ipython_2007}, and Jupyter\cite{kluyver_jupyter_2016}.
% References
\bibliography{wfirst} % bibliography data in report.bib
\bibliographystyle{spiebib} % makes bibtex use spiebib.bst

\end{document} 
